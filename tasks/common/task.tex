\documentclass[addpoints]{exam}
\usepackage[utf8]{inputenc}
\usepackage[russian]{babel}
\usepackage[OT1]{fontenc}
\usepackage{amsmath}
\usepackage{amsfonts}
\usepackage{amssymb}
\usepackage{graphicx}
\title{Неразобранные задачи}
\author{Минский ШАД. Осень}

\usepackage{tikz}
\usetikzlibrary{arrows}

\DeclareMathOperator{\prev}{prev}

\usepackage{xstring}
\usepackage{ifthen}
\usepackage{color}
\usepackage{hyperref}

\newcommand{\scorenameraw}[1]{%
    \StrRight{#1}{1}[\lastdigit]%
\ifthenelse{\equal{#1}{11}}{баллов}{\ifthenelse{\equal{#1}{12}}{баллов}{\ifthenelse{\equal{#1}{13}}{баллов}{\ifthenelse{\equal{#1}{14}}{баллов}{\ifthenelse{\equal{\lastdigit}{1}}{балл}{\ifthenelse{\equal{\lastdigit}{2}}{балла}{\ifthenelse{\equal{\lastdigit}{3}}{балла}{\ifthenelse{\equal{\lastdigit}{4}}{балла}
{\ifthenelse{\equal{\lastdigit}{$}}{балла}
{баллов}}}}}}}}}}

\newcommand{\scorename}[1]{\ignorespaces \scorenameraw{#1}}
\hqword{Задание}
\hpword{Баллы}
\htword{Сумма}

\bracketedpoints
\shadedsolutions
\pointname{}
\pointformat{[\thepoints\ \scorename{\thepoints}]}
\renewcommand{\solutiontitle}{\noindent\textbf{Решение:}\enspace}


\begin{document}

\printanswers
\maketitle

\begin{questions}

\section{Динамическое программирование}

\question[1] На прямой своими координатами задано $n$ точек. В этих точках расположеные гвоздики. Два гвоздика, находящихся в позициях $x_i$ и $x_j$ можно соединить ниткой длиной $|x_i - x_j|$ саженей. Необходимо натянуть нитки между гвоздями таким образом, чтоб к каждому гвоздю была присоединена как минимум одна нитка, а суммарная длина нитей была минимальна. Сложность алгоритма должна составлять $\mathcal{O}( n \log n)$.

\begin{solution}

Отсортируем все гвоздики по координате и будем считать, что они пронумерованы в порядке увеличения координаты. Очевидно, что гвоздик стоит соединять только с соседним гвоздём (иначе можно считать что рассматриваемый гвоздь соединён с промежуточным, а промежуточный~---  с изначальным соседом). Тогда введём величину $f_i$~--- ответ на задачу, если бы было задано только первых $i$ гвоздей. Тогда:

\begin{center}
\boxed{f_i = \min{(f_{i - 1}, f_{i - 2})} + |x_i - x_{i-1}|}
\end{center}

Последний гвоздь мы обязаны соединить с предпоследним. Мы выбираем из двух вариантов: первый соответствует случаю, когда мы соединяем гвоздь $i-1$ с гвоздём $i-2$, а второй~--- нет. Итого $\mathcal{O}(n \log{n})$ на сортировку и $\mathcal{O}(n)$ на вычисление ответа.

\end{solution}

\section{КМП}

\question[1] Для каждой позиции строки $S$ вычислить значение $a_i$~--- длину максимальной подстроки, которая начинается в $i$ и совпадает с некоторым суффиксом строки $S$. Решение должно иметь сложность $\mathcal{O}(n)$

\begin{solution}

Развернём строку и посчитаем префикс-функцию. Если мы развернём обратно массив, содержащий значения префикс-функций, то можно заметить, что это и есть ответ на задачу.

\end{solution}

\section{Ad-hoc}

\question Дана матрица размером $n \times m$. Каждый элемент матрицы равен либо единице, либо нулю. Нужно преобразовать матрицу таким образом, чтоб элемент $a_{i,j}$ был равен $1$ тогда и только тогда, когда в строке $i$ есть хотя бы одна единица или в столбце $j$ есть хотя бы одна единица. 

\begin{parts}

\part[1] Решение должно иметь сложность $\mathcal{O}(nm)$
\part[1] Решение должно иметь сложность $\mathcal{O}(nm)$ и использовать лишь константу дополнительной памяти (т.е. результат должен оказаться в исходной матрице). Каждый элемент матрицы занимает один бит.

\end{parts}

\begin{solution}

Заметим, что задачу можно переформулировать так: если в позиции $(i,j)$ исходной матрицы стоит единица, то надо заполнить строку $i$ и столбец $j$ единицами в результирующей матрице.

Изначально запомним, есть ли в первой строке хотя бы одна единица. Затем для каждой строки, начиная со второй, будем делать следующее:

\begin{enumerate}

\item \label{itm:remember_one} Запомним есть ли в этой строке хотя бы одна единица. Эту информацию не будем запоминать между строками, так что памяти будет $\mathcal{O}(1)$

\item Если в столбце $j$ этой строки стоит единица, поставим единицы в $j$-й столбец первой строки

\item Если в пункте \ref{itm:remember_one} мы запомнили, что в этой строке была единица, то заполним всю строку единицами

\end{enumerate}

Теперь пройдёмся по первой строке и если встречаем единицу, то заполняем весь встреченный столбец единицами. Если мы изначально запомнили, что первая строка содеражала хотя бы одну единицу, то заполняем единицами всю строку. Полученная матрица~--- искомая.

\end{solution}

\question[\half] Дан массив, где \textbf{к}аждое число, кроме одного, повторяет\textbf{с}я два раза, а одно число~--- встречается только \textbf{о}дин раз. Надо найти это число за $1$ п\textbf{р}оход по массиву и $\mathcal{O}(1)$ дополнительной памяти.

\begin{solution}

Найдём $\oplus$-сумму всего массива~--- это и будет искомое число.

\end{solution}

\question[1 \half] Дан массив целых чисел, где каждое число, кроме $x$ и $y$, встречается по два раза, а числа $x$ и $y$~--- ровно по одному ($x \neq y$). Надо найти эти числа за $\mathcal{O}(n)$ времени и $\mathcal{O}(1)$ памяти.

\begin{solution}

Найдём $\oplus$-сумму всего массива~--- это будет $x \oplus y$. Обозначим эту сумму за $S$. Очевидно, что $S \neq 0$ ($x \neq y$). Найдём любой его единичный бит $i$ (позже покажем, как это сделать за константу времени). Мы знаем, что в этом бите числа $x$ и $y$ различаются. Будем считать, не теряя общности, что $x$ имеет $1$ в бите $i$, а $y$~--- $0$. Тогда найдём $S_0$~--- $\oplus$-сумму всех чисел, у которых в $i$-м бите стоит $0$, и аналогичную $S_1$~--- для всех чисел, у которых в $i$-м бите стоит единица. Тогда $x = S_1$, $y = S_0$. 

Научимся находить единичный бит у числа за $\mathcal{O}(1)$. Для этого заметим, что $\prev(x) = x \& (x - 1)$~--- это число $x$ с занулённым младшим единичным битом. Тогда, если мы вычислим $x \oplus \prev(x)$, то мы как раз получим число с одним взведённым битом~--- самым младшим единичным битом числа $x$.

Затем, для определения куда отнести число $z$: в $S_0$ или $S_1$ необходимо просто проверять результат $z \& (x \oplus \prev(x))$.

\end{solution}


\section{Геометрия}

\question Дано $n$ точек на плоскости. Необходимо сказать сколько треугольников на этих точках содержат точку $(0,0)$. 

\begin{parts}
\part[\half] Решение должно иметь сложность $\mathcal{O}(n^3)$

\part[\half] Решение должно иметь сложность $\mathcal{O}(n^2 \log{n})$

\part[1] Решение должно иметь сложность $\mathcal{O}(n \log{n})$
\end{parts}

\begin{solution}

Посчитаем количество треугольников, которые \textbf{не} содержат точку $(0,0)$, а затем вычтем из общего ($C_n^3$) количества треугольников найденное количество и получим ответ на задачу.

Все такие треугольники содержатся в одной полуплоскости относительно начала координат. Отсортируем все точки по углу, относительно $(0,0)$. Начнём с полуплоскости, полученной осью $oX$. Будем вращать её по часовой стрелке. Когда точка $A$ входит в полуплоскость, посчитаем сколько треугольников будет содержаться в этой новой полуплоскости и иметь $A$ как одну из вершин. Очевидно, что если в полуплоскости ровно $m$ точек, то количество искомых треугольников $C_m^2$. Число $m$ можно легко поддерживать: при вхождении точки увеличиваем его на $1$, при выходе~--- уменьшаем.

\end{solution}

\question[3] Дан массив из $n+1$ числа, в котором содержатся целые числа от $1$ до $n$ (какие-то числа могут отсутствовать). Необходимо найти любое такое $x$, что $x$ встречается в массиве как минимум дважды.

\begin{solution}

Рассмотрим граф из $n+1$ вершины, и дугами $i \rightarrow a_i$. В таком орграфе $n+1$ вершина и $n+1$ дуга, а значит он состоит только из циклов, каждая вершина которого~--- корень какого-нибудь корневого дерева.

Например, исходный массив:

\begin{center}

\begin{tabular}{|c|c|c|c|c|c|c|c|c|}
\hline 
$a_i$ & 3 & 1 & 2 & 1 & 4 & 3 & 7 & 4 \\ 
\hline 
$i$ & 1 & 2 & 3 & 4 & 5 & 6 & 7 & 8 \\ 
\hline 
\end{tabular} 

\end{center}

Породит следующий граф:

\begin{center}

\begin{tikzpicture}[->,>=stealth',shorten >=1pt,auto,node distance=2cm,
  thick,main node/.style={circle,fill=blue!20,draw,font=\sffamily\Large\bfseries}]

  \node[main node, fill=green!] (1) {1};
  \node[main node] (2) [right of=1] {2};
  \node[main node] (3) [right of=2] {3};
  \node[main node, fill=green!] (4) [right of=3] {4};
  \node[main node] (5) [right of=4] {5};
  \node[main node] (6) [right of=5] {6};
  \node[main node] (7) [right of=6] {7};
  \node[main node] (8) [right of=7] {8};	

  \path[every node/.style={font=\sffamily\small}]
    (1) edge [bend right] node[left] {} (3)
    (2) edge node [left] {} (1)
    (3) edge node [left] {} (2)
    (4) edge [bend right] node[right] {} (1)
	(5) edge [left] node {} (4)
	(6) edge [loop below] node {} (6)
	(7) edge [bend right] node {} (4)
	(8) edge [bend left] node {} (4)
       ;
\end{tikzpicture}

\end{center}

Зелёным отмечены числа, которые подходят под ответ. Эти числа соответствуют вершинам, в которые входят больше одной дуги.

Забудем на время про ориентацию дуг. Посмотрим на компоненту связности, в которой лежит вершина $n+1$. Очевидно, что в этой компоненте есть цикл (в компоненте $m$ вершин и $m$ рёбер). Однако, вершина $n+1$ на цикле лежать не может (в неё не входит ни одна дуга)~--- значит если мы встанём в неё и будем идти по дугам, то рано или поздно придём в цикл. Заметим, что первая вершина цикла, в которую мы попадём обязательно будет <<зелёной>>. Действительно, в неё входит как минимум одна дуга из цикла и та дуга, по которой мы пришли.

Ну, теперь задача получилась простая. Надо встать в вершину $n+1$ и идти по дугам до цикла, а после найти первую вершину цикла. В данном случае можно сделать это так.

\begin{enumerate}

\item Сделаем от вершины $n+1$ ровно $n+1$ шаг. Пусть мы попали в вершину $x$. Очевидно, это вершина цикла (предпериод не может быть длинней $n+1$).

\item Пойдём от вершины $x$ по дугам и будем считать количество шагов, пока опять не попадём в вершину $x$. Пусть это количество $l$. Заметим, что $l$~--- длина цикла. 

\item Заведём два указателя. Один будет указывать на вершину $n+1$, другой на вершину через $l$ шагов от вершины $n+1$. Будем двигать эти указатели одновременно по шагу, пока они не станут указывать на одну и ту же вершину. Очевидно, что это и будет первая вершина цикла.

\end{enumerate}

\end{solution}

\end{questions}

\end{document}