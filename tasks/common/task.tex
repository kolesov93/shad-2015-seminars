\documentclass[addpoints]{exam}
\usepackage[utf8]{inputenc}
\usepackage[russian]{babel}
\usepackage[OT1]{fontenc}
\usepackage{amsmath}
\usepackage{amsfonts}
\usepackage{amssymb}
\usepackage{graphicx}
\title{Неразобранные задачи}
\author{Минский ШАД. Осень}

../../templates/tasks/exam_russification.tex

\begin{document}

\printanswers
\maketitle

\begin{questions}

\section{Динамическое программирование}

\question[1] На прямой своими координатами задано $n$ точек. В этих точках расположеные гвоздики. Два гвоздика, находящихся в позициях $x_i$ и $x_j$ можно соединить ниткой длиной $|x_i - x_j|$ саженей. Необходимо натянуть нитки между гвоздями таким образом, чтоб к каждому гвоздю была присоединена как минимум одна нитка, а суммарная длина нитей была минимальна. Сложность алгоритма должна составлять $\mathcal{O}( n \log n)$.

\begin{solution}
Отсортируем все гвоздики по координате и будем считать, что они пронумерованы в порядке увеличения координаты. Очевидно, что гвоздик стоит соединять только с соседним гвоздём (иначе можно считать что рассматриваемый гвоздь соединён с промежуточным, а промежуточный ~-- с изначальным соседом). Тогда введём величину $f_i$ ~--- ответ на задачу, если бы было задано только первых $i$ гвоздей. Тогда:

$$f_i = \min{(f_{i - 1}, f_{i - 2})} + |x_i - x_{i-1}|$$

Последний гвоздь мы обязаны соединить с предпоследним. Мы выбираем из двух вариантов: первый соответствует случаю, когда мы соединяем гвоздь $i-1$ с гвоздём $i-2$, а второй ~--- нет. Итого $\mathcal{O}(n \log{n})$ на сортировку и $\mathcal{O}(n)$ на вычисление ответа.

\end{solution}

\section{КМП}

\question[1] Для каждой позиции строки $S$ вычислить значение $a_i$ ~--- длину максимальной подстроки, которая начинается в $i$ и совпадает с некоторым суффиксом строки $S$. Решение должно иметь сложность $\mathcal{O}(n)$

\begin{solution}

Развернём строку и посчитаем префикс-функцию. Если мы развернём обратно массив, содержащий значения префикс-функций, то можно заметить, что это и есть ответ на задачу.

\end{solution}


\end{questions}



\end{document}