\documentclass[addpoints]{exam}
\usepackage[utf8]{inputenc}
\usepackage[russian]{babel}
\usepackage[OT1]{fontenc}
\usepackage{amsmath}
\usepackage{amsfonts}
\usepackage{amssymb}
\usepackage{graphicx}
\title{Неразобранные задачи}
\author{Минский ШАД. Осень}

../../tasks/common/preamble.tex

\begin{document}

%\printanswers
\maketitle

\begin{questions}

\section{Динамическое программирование}


\section{Разное}

\question[3] Дан массив из $n+1$ числа, в котором содержатся целые числа от $1$ до $n$ (какие-то числа могут отсутствовать). Необходимо найти любое такое $x$, что $x$ встречается в массиве как минимум дважды.

\begin{solution}

Рассмотрим граф из $n+1$ вершины, и дугами $i \rightarrow a_i$. В таком орграфе $n+1$ вершина и $n+1$ дуга, а значит он состоит только из циклов, каждая вершина которого~--- корень какого-нибудь корневого дерева.

Например, исходный массив:

\begin{center}

\begin{tabular}{|c|c|c|c|c|c|c|c|c|}
\hline 
$a_i$ & 3 & 1 & 2 & 1 & 4 & 3 & 7 & 4 \\ 
\hline 
$i$ & 1 & 2 & 3 & 4 & 5 & 6 & 7 & 8 \\ 
\hline 
\end{tabular} 

\end{center}

Породит следующий граф:

\begin{center}

\begin{tikzpicture}[->,>=stealth',shorten >=1pt,auto,node distance=2cm,
  thick,main node/.style={circle,fill=blue!20,draw,font=\sffamily\Large\bfseries}]

  \node[main node, fill=green!] (1) {1};
  \node[main node] (2) [right of=1] {2};
  \node[main node] (3) [right of=2] {3};
  \node[main node, fill=green!] (4) [right of=3] {4};
  \node[main node] (5) [right of=4] {5};
  \node[main node] (6) [right of=5] {6};
  \node[main node] (7) [right of=6] {7};
  \node[main node] (8) [right of=7] {8};	

  \path[every node/.style={font=\sffamily\small}]
    (1) edge [bend right] node[left] {} (3)
    (2) edge node [left] {} (1)
    (3) edge node [left] {} (2)
    (4) edge [bend right] node[right] {} (1)
	(5) edge [left] node {} (4)
	(6) edge [loop below] node {} (6)
	(7) edge [bend right] node {} (4)
	(8) edge [bend left] node {} (4)
       ;
\end{tikzpicture}

\end{center}

Зелёным отмечены числа, которые подходят под ответ. Эти числа соответствуют вершинам, в которые входят больше одной дуги.

Забудем на время про ориентацию дуг. Посмотрим на компоненту связности, в которой лежит вершина $n+1$. Очевидно, что в этой компоненте есть цикл (в компоненте $m$ вершин и $m$ рёбер). Однако, вершина $n+1$ на цикле лежать не может (в неё не входит ни одна дуга)~--- значит если мы встанём в неё и будем идти по дугам, то рано или поздно придём в цикл. Заметим, что первая вершина цикла, в которую мы попадём обязательно будет <<зелёной>>. Действительно, в неё входит как минимум одна дуга из цикла и та дуга, по которой мы пришли.

Ну, теперь задача получилась простая. Надо встать в вершину $n+1$ и идти по дугам до цикла, а после найти первую вершину цикла. В данном случае можно сделать это так.

\begin{enumerate}

\item Сделаем от вершины $n+1$ ровно $n+1$ шаг. Пусть мы попали в вершину $x$. Очевидно, это вершина цикла (предпериод не может быть длинней $n+1$).

\item Пойдём от вершины $x$ по дугам и будем считать количество шагов, пока опять не попадём в вершину $x$. Пусть это количество $l$. Заметим, что $l$~--- длина цикла. 

\item Заведём два указателя. Один будет указывать на вершину $n+1$, другой на вершину через $l$ шагов от вершины $n+1$. Будем двигать эти указатели одновременно по шагу, пока они не станут указывать на одну и ту же вершину. Очевидно, что это и будет первая вершина цикла.

\end{enumerate}

\end{solution}




\question Дана матрица размером $n \times m$. Каждый элемент матрицы равен либо единице, либо нулю. Нужно преобразовать матрицу таким образом, чтоб элемент $a_{i,j}$ был равен $1$ тогда и только тогда, когда в строке $i$ есть хотя бы одна единица или в столбце $j$ есть хотя бы одна единица. 

\begin{parts}

\part[1] Решение должно иметь сложность $\mathcal{O}(nm)$
\part[1] Решение должно иметь сложность $\mathcal{O}(nm)$ и использовать лишь константу дополнительной памяти (т.е. результат должен оказаться в исходной матрице). Каждый элемент матрицы занимает один бит.

\end{parts}

\begin{solution}

Заметим, что задачу можно переформулировать так: если в позиции $(i,j)$ исходной матрицы стоит единица, то надо заполнить строку $i$ и столбец $j$ единицами в результирующей матрице.

Изначально запомним, есть ли в первой строке хотя бы одна единица. Затем для каждой строки, начиная со второй, будем делать следующее:

\begin{enumerate}

\item \label{itm:remember_one} Запомним есть ли в этой строке хотя бы одна единица. Эту информацию не будем запоминать между строками, так что памяти будет $\mathcal{O}(1)$

\item Если в столбце $j$ этой строки стоит единица, поставим единицы в $j$-й столбец первой строки

\item Если в пункте \ref{itm:remember_one} мы запомнили, что в этой строке была единица, то заполним всю строку единицами

\end{enumerate}

Теперь пройдёмся по первой строке и если встречаем единицу, то заполняем весь встреченный столбец единицами. Если мы изначально запомнили, что первая строка содеражала хотя бы одну единицу, то заполняем единицами всю строку. Полученная матрица~--- искомая.

\end{solution}

\question[\half] Дан массив, где \textbf{к}аждое число, кроме одного, повторяет\textbf{с}я два раза, а одно число~--- встречается только \textbf{о}дин раз. Надо найти это число за $1$ п\textbf{р}оход по массиву и $\mathcal{O}(1)$ дополнительной памяти.

\begin{solution}

Найдём $\oplus$-сумму всего массива~--- это и будет искомое число.

\end{solution}

\question[1 \half] Дан массив целых чисел, где каждое число, кроме $x$ и $y$, встречается по два раза, а числа $x$ и $y$~--- ровно по одному ($x \neq y$). Надо найти эти числа за $\mathcal{O}(n)$ времени и $\mathcal{O}(1)$ памяти.

\begin{solution}

Найдём $\oplus$-сумму всего массива~--- это будет $x \oplus y$. Обозначим эту сумму за $S$. Очевидно, что $S \neq 0$ ($x \neq y$). Найдём любой его единичный бит $i$ (позже покажем, как это сделать за константу времени). Мы знаем, что в этом бите числа $x$ и $y$ различаются. Будем считать, не теряя общности, что $x$ имеет $1$ в бите $i$, а $y$~--- $0$. Тогда найдём $S_0$~--- $\oplus$-сумму всех чисел, у которых в $i$-м бите стоит $0$, и аналогичную $S_1$~--- для всех чисел, у которых в $i$-м бите стоит единица. Тогда $x = S_1$, $y = S_0$. 

Научимся находить единичный бит у числа за $\mathcal{O}(1)$. Для этого заметим, что $\prev(x) = x \& (x - 1)$~--- это число $x$ с занулённым младшим единичным битом. Тогда, если мы вычислим $x \oplus \prev(x)$, то мы как раз получим число с одним взведённым битом~--- самым младшим единичным битом числа $x$.

Затем, для определения куда отнести число $z$: в $S_0$ или $S_1$ необходимо просто проверять результат $z \& (x \oplus \prev(x))$.

\end{solution}


\section{Геометрия}

\question Дано $n$ точек на плоскости. Необходимо сказать сколько треугольников на этих точках содержат точку $(0,0)$. 

\begin{parts}
\part[\half] Решение должно иметь сложность $\mathcal{O}(n^3)$

\part[\half] Решение должно иметь сложность $\mathcal{O}(n^2 \log{n})$

\part[1] Решение должно иметь сложность $\mathcal{O}(n \log{n})$
\end{parts}

\begin{solution}

Посчитаем количество треугольников, которые \textbf{не} содержат точку $(0,0)$, а затем вычтем из общего ($C_n^3$) количества треугольников найденное количество и получим ответ на задачу.

Все такие треугольники содержатся в одной полуплоскости относительно начала координат. Отсортируем все точки по углу, относительно $(0,0)$. Начнём с полуплоскости, полученной осью $oX$. Будем вращать её по часовой стрелке. Когда точка $A$ входит в полуплоскость, посчитаем сколько треугольников будет содержаться в этой новой полуплоскости и иметь $A$ как одну из вершин. Очевидно, что если в полуплоскости ровно $m$ точек, то количество искомых треугольников $C_m^2$. Число $m$ можно легко поддерживать: при вхождении точки увеличиваем его на $1$, при выходе~--- уменьшаем.

\end{solution}


\section{Структуры данных}

\question Предложить реализацию очереди, используя стурктуру данных стек. Разрешается использовать $\mathcal{O}(1)$ стеков и $\mathcal{O}(1)$ дополнительной памяти. Стек имеет две операции (<<push>> и <<pop>>), очередь тоже.

\begin{parts}

\part[\half] Амортизированная стоимость операций должна быть $\O{1}$

\part[2] Стоимость операций должна быть $\O{1}$ в худшем случае

\end{parts}

\begin{solution}

Для решения первой части можно использовать известную конструкцию из двух стеков, назовём их $A$ и $B$. При выполнении операции <<push>> будем добавлять элемент в стек $B$. При выполнении <<pull>>~--- достаем из стека $A$. Если же стек $A$ пуст, то просто перекладываем все элементы из стека $B$ в стек $A$. Легко показать, что тогда выполняется FIFO. Каждый элемент не более одного раза добавляется и достаётся из каждого из двух стеков.

Решение второго пункта довольно большое, описание его можно найти вот тут \url{http://goo.gl/VjEYxL}

\end{solution}

\question Предложить реализацию стека, используя стурктуру данных очередь. Стек имеет две операции (<<push>> и <<pop>>), очередь тоже. За $n$ будем считать максимальное количество элементов, которые могут находится в стеке в одно время.

\begin{parts}

\part[\half] Разрешается использовать $\O{n}$ дополнительной памяти

\part[\half] Стоимость операции <<push>> должна быть $\O{n}$, а <<pop>>~--- $\O{1}$. Дополнительной памяти~--- $\O{1}$

\part[\half] Стоимость операции <<pop>> должна быть $\O{n}$, а <<push>>~--- $\O{1}$

\part[2] (Амортизированная) Стоимость обеих операций должна быть $\o{n}$

\end{parts}

\begin{solution}

В первом пункте просто заведём стек в дополнительной памяти.

Будем реализовывать стек на одной очереди. Будем поддерживать в этой очереди порядок LIFO. Если надо забрать элемент, то просто заберём элемент из очереди. Если вставить, то поступим следующим образом. Запомним сколько элементов в очереди до добавления, пусть $m$. Добавим новый элемент в очередь. Затем сделаем $m$ раз связку <<pull>>-<<push>>, тем самым выведя новый элемент на первое место, а все остальные оставив за ним. На рисунке элементы пронумерованы в порядке добавления в структуру.

\begin{center}
\begin{tabular}{|c|c|}
\hline 
Промежуточное состояние структуры &  
\begin{tikzpicture}[->,>=stealth',shorten >=1pt,auto,node distance=2cm,
  thick,main node/.style={fill=white!20,draw,font=\sffamily\Large\bfseries}]

  \node[main node] (1) {4};
  \node[main node] (2) [right of=1] {3};
  \node[main node] (3) [right of=2] {2};
  \node[main node] (4) [right of=3] {1};
  
  \path[every node/.style={font=\sffamily\small}]
    (2) edge node [left] {} (1)
    (3) edge node [left] {} (2)
    (4) edge node [left] {} (3)
       ;
\end{tikzpicture}
\\ 
\hline 
<<pull>> & 
\begin{tikzpicture}[->,>=stealth',shorten >=1pt,auto,node distance=2cm,
  thick,main node/.style={fill=white!20,draw,font=\sffamily\Large\bfseries}]

  \node[main node, -, cross out, draw=red] (1) {4};
  \node[main node] (2) [right of=1] {3};
  \node[main node] (3) [right of=2] {2};
  \node[main node] (4) [right of=3] {1};
  
  \path[every node/.style={font=\sffamily\small}]
    (3) edge node [left] {} (2)
    (4) edge node [left] {} (3)
       ;
\end{tikzpicture}
\\
\hline
После <<pull>> & 
\begin{tikzpicture}[->,>=stealth',shorten >=1pt,auto,node distance=2cm,
  thick,main node/.style={fill=white!20,draw,font=\sffamily\Large\bfseries}]

  \node[main node] (2) [right of=1] {4};
  \node[main node] (3) [right of=2] {3};
  \node[main node] (4) [right of=3] {2};
  
  \path[every node/.style={font=\sffamily\small}]
    (3) edge node [left] {} (2)
    (4) edge node [left] {} (3)
       ;
\end{tikzpicture}
\\
\hline
Добавление элемента & 
\begin{tikzpicture}[->,>=stealth',shorten >=1pt,auto,node distance=2cm,
  thick,main node/.style={fill=white!20,draw,font=\sffamily\Large\bfseries}]

  \node[main node] (2) [] {4};
  \node[main node] (3) [right of=2] {3};
  \node[main node] (4) [right of=3] {2};
  \node[main node, fill=green!] (5) [right of=4] {5};
  
  \path[every node/.style={font=\sffamily\small}]
    (3) edge node [left] {} (2)
    (4) edge node [left] {} (3)
    (5) edge node [left] {} (4)
       ;
\end{tikzpicture}
\\
\hline
<<pull>>-<<push>> & 
\begin{tikzpicture}[->,>=stealth',shorten >=1pt,auto,node distance=2cm,
  thick,main node/.style={fill=white!20,draw,font=\sffamily\Large\bfseries}]

  \node[main node] (3) [] {3};
  \node[main node] (4) [right of=3] {2};
  \node[main node, fill=green!] (5) [right of=4] {5};
  \node[main node] (2) [right of=5] {4};  
  
  \path[every node/.style={font=\sffamily\small}]
    (4) edge node [left] {} (3)
    (5) edge node [left] {} (4)
    (2) edge node [left] {} (5)
       ;
\end{tikzpicture}
\\
\hline  
<<pull>>-<<push>> & 
\begin{tikzpicture}[->,>=stealth',shorten >=1pt,auto,node distance=2cm,
  thick,main node/.style={fill=white!20,draw,font=\sffamily\Large\bfseries}]

  \node[main node] (4) [] {2};
  \node[main node, fill=green!] (5) [right of=4] {5};
  \node[main node] (2) [right of=5] {4};  
  \node[main node] (3) [right of=2] {3};  
  
  \path[every node/.style={font=\sffamily\small}]
    (3) edge node [left] {} (2)
    (5) edge node [left] {} (4)
    (2) edge node [left] {} (5)
       ;
\end{tikzpicture}
\\
\hline
<<pull>>-<<push>> & 
\begin{tikzpicture}[->,>=stealth',shorten >=1pt,auto,node distance=2cm,
  thick,main node/.style={fill=white!20,draw,font=\sffamily\Large\bfseries}]

  \node[main node, fill=green!] (5) [] {5};
  \node[main node] (2) [right of=5] {4};  
  \node[main node] (3) [right of=2] {3};  
  \node[main node] (4) [right of=3] {2};
  
  \path[every node/.style={font=\sffamily\small}]
    (3) edge node [left] {} (2)
    (2) edge node [left] {} (5)
    (4) edge node [left] {} (3)
       ;
\end{tikzpicture}
\\
\hline
\end{tabular} 
\end{center}

Понятно, что можно при операции <<push>> просто добавлять элемент в очередь, а при <<pop>> делать серию <<pull>>-<<push>> дабы вывести самый последний элемент на первое место.

Приступим к интересной части. Реализуем операцию <<push>> за $\o{1}$ и <<pull>> за амортизированную оценку $\O{sqrt{n}}$. Заведём две очереди: $A$ и $B$. В первой будем хранить $\approx \sqrt{n}$ элементов самых близких к вершине стека (т.е. добавленных позже всего). Хранить их, однако, будем в порядке очереди (иначе сложность, вероятно, бы возрасла опять до $\O(n)$). При операции <<push>> просто добавляем элемент в очередь $A$. Если $|A|^2 > |B|$ перекидываем элемент верхней элемент очереди $A$ в $B$. Понятно, что таких перекидываний на <<push>> надо сделать не больше одного.

При операции <<pull>> серией <<pull>>-<<push>> очереди $B$ добиваемся, чтобы последний элемент стека оказался на вершине и возвращаем его. Заметим, что инвариант про $|A|^2 \leqslant |B|$ не нарушился.

Если же очередь $A$ пуста, то необходимо переупорядочить элементы. Для этого переместим (B.pull~-A.push) $\floor{|B| - \sqrt{|B|}}$ элементов из очереди $B$ в очередь $A$ и просто переименуем очереди. Теперь инвариант выполняется.

В качестве упражнения предлагается доказать, что амортизированная сложность операции <<pull>> составляет $\sqrt{n}$.


\end{solution}

\question[1 \half] Дан массив из $n = 2^k$ различных целых чисел. Необходимо определить 2-ю порядковую статистику за не более, чем $n + k - 2$ сравнения.

\begin{solution}

\tikzset{
  treenode/.style = {align=center, inner sep=0pt, text centered,
    font=\sffamily, circle, fill=white, font=\sffamily\bfseries, draw,
     text width=1.5em},
  best/.style = {treenode,  fill=green},
  prebest/.style = {treenode, fill=yellow},
  cand/.style = {treenode, fill=blue},
}

Заметим, что найти минимум в массиве из $n$ элементов можно только с помощью $n-1$ сравнения, причём, очевидно, меньше сделать нельзя.

С другой стороны можно по разному использовать эти сравнения. Давайте будем поступать следующим образом. На первом шаге сравним элементы на первом и втором местах, затем на третьем и четвёртом и так за $n/2$ сравнений оставим ровно $n/2$ кандидатов на минимум. Будем повторять такую операцию, пока не останется ровно один элемент. Такую стратегию легко реализовать в виде дерева. К примеру, рассмотрим массив $(2,8,1,3,7,4,6,5)$:

\begin{center}
\begin{tikzpicture}[<-,>=stealth',level/.style={sibling distance = 5cm/#1,
  level distance = 1.5cm}] 
\node [treenode] {1}
    child{ node [treenode] {1} 
            child{ node [treenode] {2}
              child { node[treenode] {2}}
              child { node[treenode] {8}}
            }
            child{ node [treenode] {1}
              child { node[treenode] {1}}
              child { node[treenode] {3}}
            }                             
    }
    child{ node [treenode] {4} 
            child{ node [treenode] {4}
              child { node[treenode] {7}}
              child { node[treenode] {4}}
            }
            child{ node [treenode] {5}
              child { node[treenode] {6}}
              child { node[treenode] {5}}
            }                             
    }
; 
\end{tikzpicture}
\end{center}

Посмотрим, как минимум проложил себе путь наверх:

\begin{center}
\begin{tikzpicture}[<-,>=stealth',level/.style={sibling distance = 5cm/#1,
  level distance = 1.5cm}] 
\node [best] {1}
    child{ node [best] {1} 
            child{ node [treenode] {2}
              child { node[treenode] {2}}
              child { node[treenode] {8}}
            }
            child{ node [best] {1}
              child { node[best] {1}}
              child { node[treenode] {3}}
            }                             
    }
    child{ node [treenode] {4} 
            child{ node [treenode] {4}
              child { node[treenode] {7}}
              child { node[treenode] {4}}
            }
            child{ node [treenode] {5}
              child { node[treenode] {6}}
              child { node[treenode] {5}}
            }                             
    }
; 

\end{tikzpicture}
\end{center}

Очевидно, он выигрывал при каждом сравнении. Теперь заметим, что среди тех, у кого он выигрывал обязательно есть второй минимум. Действительно, второй минимум таким же образом шёл наверх, выигрывая всех, пока не встретился с минимумом:

\begin{center}
\begin{tikzpicture}[<-,>=stealth',level/.style={sibling distance = 5cm/#1,
  level distance = 1.5cm}] 
\node [best] {1}
    child{ node [best] {1} 
            child{ node [prebest] {2}
              child { node[prebest] {2}}
              child { node[treenode] {8}}
            }
            child{ node [best] {1}
              child { node[best] {1}}
              child { node[treenode] {3}}
            }                             
    }
    child{ node [treenode] {4} 
            child{ node [treenode] {4}
              child { node[treenode] {7}}
              child { node[treenode] {4}}
            }
            child{ node [treenode] {5}
              child { node[treenode] {6}}
              child { node[treenode] {5}}
            }                             
    }
; 

\end{tikzpicture}
\end{center}

Заметим, что всего элементов, которых мы сравнивали с минимумом ровно $k$~--- по одному на каждый уровень:

\begin{center}
\begin{tikzpicture}[<-,>=stealth',level/.style={sibling distance = 5cm/#1,
  level distance = 1.5cm}] 
\node [best] {1}
    child{ node [best] {1} 
            child{ node [cand] {2}
              child { node[treenode] {2}}
              child { node[treenode] {8}}
            }
            child{ node [best] {1}
              child { node[best] {1}}
              child { node[cand] {3}}
            }                             
    }
    child{ node [cand] {4} 
            child{ node [treenode] {4}
              child { node[treenode] {7}}
              child { node[treenode] {4}}
            }
            child{ node [treenode] {5}
              child { node[treenode] {6}}
              child { node[treenode] {5}}
            }                             
    }
; 

\end{tikzpicture}
\end{center}

А значит, из них мы можем найти минимум за $k-1$ сравнение. Таким образом нам нужно $n-1+k-1=n+k-2$ сравнения на всё.

\end{solution}


\end{questions}

\end{document}
