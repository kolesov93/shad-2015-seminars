\documentclass{beamer}
\usepackage[utf8]{inputenc}
\usepackage[russian]{babel}
\usetheme{Warsaw}
\subtitle[Алгоритмы и структуры данных поиска]{Алгоритмы и структуры данных поиска}
\title[Pattern matching]{Pattern matching}
\author{Алексей Колесов}
\institute{ШАД 2015. Минск}
\date{26 декабря, 2014}

\theoremstyle{definition}
\newtheorem{mydef}[theorem]{Определение}

\usepackage{listings}
\usepackage{xcolor, soul, tikz, verbatim}
\usepackage{lstlinebgrd}
\lstset { %
    language=C++,
    backgroundcolor=\color{blue!5}, % set backgroundcolor
    basicstyle=\footnotesize,% basic font setting
    numbers=left,
    frame=single,
    captionpos=b,
    keywordstyle=\color{blue}\textbf,
    morekeywords={vector, string, size_t},
    escapeinside={(*@}{@*)},
    moredelim=**[is][\btHL]{`}{`}
}
\renewcommand{\lstlistingname}{Листинг}

%\setbeameroption{show notes}

\makeatletter
\newenvironment{btHighlight}[1][]
{\begingroup\tikzset{bt@Highlight@par/.style={#1}}\begin{lrbox}{\@tempboxa}}
{\end{lrbox}\bt@HL@box[bt@Highlight@par]{\@tempboxa}\endgroup}

\newcommand\btHL[1][]{%
  \begin{btHighlight}[#1]\bgroup\aftergroup\bt@HL@endenv%
}
\def\bt@HL@endenv{%
  \end{btHighlight}%   
  \egroup
}
\newcommand{\bt@HL@box}[2][]{%
  \tikz[#1]{%
    \pgfpathrectangle{\pgfpoint{1pt}{0pt}}{\pgfpoint{\wd #2}{\ht #2}}%
    \pgfusepath{use as bounding box}%
    \node[anchor=base west, fill=orange!30,outer sep=0pt,inner xsep=1pt, inner ysep=0pt, rounded corners=3pt, minimum height=\ht\strutbox+1pt,#1]{\raisebox{1pt}{\strut}\strut\usebox{#2}};
  }%
}
\makeatother

\begin{document}

\begin{frame}
\titlepage
\end{frame}

\begin{frame}
\frametitle{Содержание}
\tableofcontents[currentsection]
\end{frame}

\section{z-алгоритм}
\subsection{Определения}

\begin{frame}{z-блок}

\begin{mydef}
Будем называть \alert{z-блоком} пару $(l, r)$ в случае если подстрока $s[l \ldots r)$ совпадает с префиксом строки $s$. 
\end{mydef}

\begin{exampleblock} {Примеры z-блоков}
\begin{center}
\begin{tabular}{|c|c|c|c|c|c|c|}
\hline 
a & b & a & c & a & b & a \\ 
\hline 
0 & 1 & 2 & 3 & 4 & 5 & 6 \\ 
\hline 
\end{tabular} 
\end{center}

Например, для строки \texttt{<<abacaba>>} z-блоками будут являться пары $(0, 5)$, $(4,7)$, $(2,3)$ и другие.
\end{exampleblock}

\end{frame}

\begin{frame}{z-функция}

\begin{mydef}
\alert{z-функция} $z[i]$ ~--- длина максимального $z$-блока, который начинается в символе $i$. 
\end{mydef}

\begin{alertblock}{Нулевой символ}
Согласно определению, $z[0] = |s|$. Однако, для удобства реализации, рассуждений и причин, коих вам пока не дано понять, будем считать, что $z[0]$ равно котику.
\end{alertblock}

\begin{exampleblock}{Пример}
\begin{center}
\begin{tabular}{|c|c|c|c|c|c|c|}
\hline 
a & b & a & c & a & b & a \\ 
\hline 
\includegraphics[scale=0.06]{cat.png} & 0 & 1 & 0 & 3 & 0 & 1 \\ 
\hline 
\end{tabular} 
\end{center}
\end{exampleblock}

\begin{center}

\end{center}

\end{frame}

\subsection{Реализация}

\begin{frame}[fragile]{Лобовая реализация}

\begin{lstlisting}[
	basicstyle=\tiny, 
	caption=Лобовая реализация Z-алгоритма]
vector<size_t> calculate_z(const string& s) {
    vector<size_t> z(s.size());  

    size_t left = 0, right = 0;
    for (size_t i = 1; i < z.size(); ++i) {
        if (right <= i) {     
            left = right = i;
            while(right < s.size() && s[right] == s[right-left]) ++right;
            z[i] = right - left;
        } else {
            if (z[i - left] < right - i) {
                z[i] = z[i - left];
                continue;
            }
            z[i] = right - i;
            while (i + z[i] < s.size() && s[i + z[i]] == s[z[i]]) {
                ++z[i];
            }
            if (i + z[i] > right) {
                left = i;
                right = i + z[i];
            }
        }
    }

    return z;
}
\end{lstlisting}

\end{frame}

\begin{frame}[fragile]{Чуть лучше}

\begin{lstlisting}[
	basicstyle=\tiny, 
	caption=Чуть улучшенная реализация Z-алгоритма]
vector<size_t> calculate_z(const string& s) {
    vector<size_t> z(s.size());

    size_t left = 0, right = 0;
    for (size_t i = 1; i < z.size(); ++i) {
        if (right <= i) {
            left = right = i;
            while(right < s.size() && s[right] == s[right-left]) ++right;
            z[i] = right - left;
        } else {
            z[i] = min(right - i, z[i - left]);
            while (i + z[i] < s.size() && s[i + z[i]] == s[z[i]]) {
                ++z[i];
            }
            if (i + z[i] > right) {
                left = i;
                right = i + z[i];
            }
        }
    }

    return z;
}

\end{lstlisting}

\end{frame}

\begin{frame}[fragile]{Приемлимая версия}

\begin{lstlisting}[
	basicstyle=\tiny, 
	caption=Реализация Z-алгоритма]
vector<size_t> calculate_z(const string& s) {
    vector<size_t> z(s.size());

    size_t left = 0, right = 0;
    for (size_t i = 1; i < z.size(); ++i) {
        if (i < right) {
            z[i] = min(right - i, z[i - left]);
        }
        while (i + z[i] < s.size() && s[i + z[i]] == s[z[i]]) ++z[i];
        if (i + z[i] > right) {
            left = i;
            right = i + z[i];
        }
    }

    return z;
}


\end{lstlisting}

\end{frame}

\end{document}
