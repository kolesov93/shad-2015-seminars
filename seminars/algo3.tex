\documentclass[addpoints]{exam}
\usepackage[utf8]{inputenc}
\usepackage[russian]{babel}
\usepackage[OT1]{fontenc}
\usepackage{amsmath}
\usepackage{amsfonts}
\usepackage{amssymb}
\usepackage{graphicx}
\usepackage{hyperref}
\title{Семинар 3}
\author{Минский ШАД. Весна}

\DeclareMathOperator{\ord}{ord}
\DeclareMathOperator{\suf}{suf}
\DeclareMathOperator{\uniq}{uniq}
\DeclareMathOperator{\cnt}{cnt}


\begin{document}

\maketitle

\section{Комбинаторика}

Даны два числа $n$ и $m$. Надо найти количество пар строк $S$ и $T$ над алфавитом размера $C$ таких, что $|S| = n$, $|T| = m$ и $S$ является подстрокой $T$.

\section{Боевой клич}

Как известно, в русском языке ровно $n$ страшных слов (их Вы, конечно, знаете; также Вы знаете, что суммарная длина всех слов равна $L$). Страшнота $i$-го страшного слова равна $w_i$ условных котиков. Надо найти самую страшную строку из $m$ символов. Страшнота строки равна сумме всех страшных слов в неё входящих (слово может входить произвольное количество раз, вхождения могут перекрываться). Сложность алгоритма должна составлять $\mathcal{O}(mL)$

\section{Различные строки}

Задана бинарная строка своей длиной ($N$) и позициями единиц (их $K$). Необходимо найти количество различных подстрок данной строки за время $\mathcal{O}(K^2 \log{K})$

\section{Задача о накачке}

Дана строка $S$ и $T$ над бинарным алфавитом. Рассмотрим следующую функцию от строки:

\begin{itemize}
\item $f(\mbox{<<0>>}) = \mbox{<<00>>}$
\item $f(\mbox{<<1>>}) = \mbox{<<01>>}$
\item $f(S) = f(S_0) + f(S_{[1:|S| - 1]})$, если $|S| > 1$, <<+>> означает конкатенацию.
\end{itemize}

Например, $f(\mbox{<<0010>>})$ = <<00000100>>. Надо найти, какое минимальное количество раз нужно применить функцию $f$ к строке $S$, чтобы строка $T$ стала подстрокой $S$ (или сказать, что это невозможно).

\end{document}
