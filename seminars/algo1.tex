\documentclass[addpoints]{exam}
\usepackage[utf8]{inputenc}
\usepackage[russian]{babel}
\usepackage[OT1]{fontenc}
\usepackage{amsmath}
\usepackage{amsfonts}
\usepackage{amssymb}
\usepackage{graphicx}
\title{Семинар 1}
\author{Минский ШАД. Весна}


\DeclareMathOperator{\ord}{ord}
\DeclareMathOperator{\suf}{suf}
\DeclareMathOperator{\uniq}{uniq}
\DeclareMathOperator{\cnt}{cnt}


\begin{document}

\maketitle

\section{Задача про наименьший период}

Дана строка $S$. Надо найти длину наименьшего её периода, т.е. такой строки $T$, что строка $S$ является префиксом строки $T^{\infty}$ (т.е. строки $T$ приписанной к самой себе бесконечное количество раз). Необходимо решение за $\mathcal{O}(n)$ с помощью полиномиального хеширования. 

Эту задачу чуть позже мы сможем решить честно за линейное время.

\section{Задача про общую подстроку}

Даны две строки $S$ и $T$, причём $|S| \leqslant |T| = n$. Необходимо найти длину наибольшей общей подстроки двух строк за время $\mathcal{O}(n \log{n})$. Требуется решение с помощью полиномиального хеширования. 

Эту задачу чуть позже мы сможем решить честно за линейное время.

\section{Задача про вхождения в интервал}

Дано $n$ строк пронумерованных от $1$ до $n$. Обозначим их суммарную длину за $N$. Также есть $m$ запросов. Каждый запрос это тройка $(l, r, s)$, причём $1 \leqslant l \leqslant r \leqslant n$, $s$~--- произвольная строка. Обозначим сумму длин всех строк из запроса за $M$. Все запросы заданы заранее. Ответ на запрос~--- количество строк с номерами от $l$ до $r$, которые содержат строку $s$ как подстроку. Необходимо решение за $\mathcal{O}(\sqrt{M} N)$.

Чуть позже мы сможем решить эту задачу честно за $\mathcal{O}((M+N)\log{M+N})$. Кто это сделает~--- получит хороший бонус.

\section{Задача на повторение}

Пусть $T(n) \leqslant T( \lceil \sqrt{n} \rceil ) + 1$. Найдите наилучшую верхнюю оценку для $T$, которую можете.

\end{document}
