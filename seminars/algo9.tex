\documentclass[addpoints]{exam}
\usepackage[utf8]{inputenc}
\usepackage[russian]{babel}
\usepackage[OT1]{fontenc}
\usepackage{amsmath}
\usepackage{amsfonts}
\usepackage{amssymb}
\usepackage{graphicx}
\usepackage{hyperref}
\title{Семинар 9}
\author{Минский ШАД. Весна}

\DeclareMathOperator{\ord}{ord}
\DeclareMathOperator{\suf}{suf}
\DeclareMathOperator{\uniq}{uniq}
\DeclareMathOperator{\cost}{cost}
\DeclareMathOperator{\suc}{suc}


\begin{document}

\maketitle

\section{Вот это поворот}

Дано $n$ точек на плоскости своими координатами. Также дано $m$ пар $(a_i, b_i)$, которые означают, что точки с номерами $a_i$ и $b_i$ соединены отрезком. Необходимо найти длину кратчайшего пути из точки $A$ в точку $B$. Ходить можно только по отрезкам, менять отрезки можно только в точках, единицу расстояния проходим за единицу времени. Во время смены отрезков тратится $C \phi$ времени, где $C$~--- наперёд заданная константа, а $\phi$~--- угол в радианах между двумя отрезками. Сложность должна составлять $\mathcal{O}(m \log{m})$

\section{Когда мы, Док?}

В недалёком прошлом было всего $n$ городов. Ваша задача была добраться из города $A$ в город $B$. У вас есть расписание поездов. Каждая запись имеет вид $(a_i,b_i,t_i,d_i)$, она означает, что поезд отправляется из города $a_i$ в город $b_i$ в момент времени $t_i$, причём его путь занимает ровно $d_i$ единиц времени. Прошу заметить, что $d_i$ может быть отрицательным. Ваша задача определить в какой минимальный момент времени можно оказаться в городе $B$, если вы начинаете свой путь в городе $A$ в момент времени 0. Сложность должна составлять $\mathcal{O}(m \log{m})$

\section{Круги своя}

Дан $(n,m)$-граф без отрицательных циклов. Необходимо найти вес самого короткого цикла за $\mathcal{O}(n^3)$

\section{Главное не победа}

Дан $(n,m)$-граф без отрицательных циклов. Найти второй по величине путь из $A$ в $B$ за $\mathcal{O}(n \log{m})$

\section{Дукат}

Дан $(n,m)$-граф. Необходимо найти количество пар рёбер таких, что одновременное удаление их из графа увеличивает количество компонент связности.

\end{document}
