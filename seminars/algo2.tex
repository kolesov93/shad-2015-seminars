\documentclass[addpoints]{exam}
\usepackage[utf8]{inputenc}
\usepackage[russian]{babel}
\usepackage[OT1]{fontenc}
\usepackage{amsmath}
\usepackage{amsfonts}
\usepackage{amssymb}
\usepackage{graphicx}
\usepackage{hyperref}
\title{Семинар 2}
\author{Минский ШАД. Весна}


\DeclareMathOperator{\ord}{ord}
\DeclareMathOperator{\suf}{suf}
\DeclareMathOperator{\uniq}{uniq}
\DeclareMathOperator{\cnt}{cnt}


\begin{document}

\maketitle

\section{Перевод префикс-функции в z-функцию}

Дана последовательность чисел $\pi$, причём известно, что существует такая строка $S$, что $\pi_i$~--- значение префикс-функции строки $S$ в позиции $i$.

Необходимо посчитать последовательность значений z-функций данной строки.

Например, вам дана последовательность $(0, 0, 1, 0, 1, 2, 3)$  (такой последовательности может соответствовать, например, строка <<abacaba>>). Тогда ответом нужно выдать последовательность $(0, 0, 1, 0, 3, 0, 1)$.

Нужно также показать, что ответ всегда однозначен.

Мы точно рассмотрим решение за $\mathcal{O}(n^2)$. Более быстрые решения рассмотрим, если у кого-то будет желание рассказать.

Существует линейный алгоритм решения данной задачи.

\section{Approximate string matching: Easy one}

Даны строка $S$ и паттерн $T$. Необходимо найти все позиции строки $S$, к которым можно приложить строку $T$ так, чтоб количество несовпавших символов не превосходило $1$. Например, если $S=\mbox{<<abacaba>>}$ и $T=\mbox{<<aba>>}$, то таких позиций $3$: $0$, $2$ и $4$ в $0$-индексации.

Время решения не должно превосходить $\mathcal{O}(|S| + |T|)$.

\section{Задача про шифр}

Даны строка $S$ и паттерн $T$, причём строка $S$ зашифрована шифром простой замены (\url{http://goo.gl/8A4MWw}). Необходимо найти все такие позиции, что паттерн мог бы совпасть с подстрокой строки $S$. Время решения $\mathcal{O}(|S| + |T|)$.

\section{Задача про странный pattern matching}

Задан бинарный массив из $N$ битов. Нам даны только $K$ чисел~--- позиции единиц в нём, остальные биты~--- нули. Также дан паттерн длиной не более чем  $M$. Надо найти количество вхождений паттерна в массив за время $\mathcal{O}(M + K \log{K})$.

\section{Задача не про pattern matching}

Даны строка $S$ и паттерн $T$. Необходимо найти количество таких позиций, что если приложить паттерн к строке, то не будет никаких совпадений.

\end{document}
