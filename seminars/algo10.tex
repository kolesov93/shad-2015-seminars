\documentclass[addpoints]{exam}
\usepackage[utf8]{inputenc}
\usepackage[russian]{babel}
\usepackage[OT1]{fontenc}
\usepackage{amsmath}
\usepackage{amsfonts}
\usepackage{amssymb}
\usepackage{graphicx}
\usepackage{hyperref}
\title{Семинар 10}
\author{Минский ШАД. Весна}

\DeclareMathOperator{\ord}{ord}
\DeclareMathOperator{\suf}{suf}
\DeclareMathOperator{\uniq}{uniq}
\DeclareMathOperator{\cost}{cost}
\DeclareMathOperator{\suc}{suc}


\begin{document}

\maketitle

\section{Гамильтонова цепь}

Найти гамильтонову цепь в ациклическом графе

\section{Граф неправильный для Дейкстры}

Нарисовать граф неправильный для Дейкстры

\section{Исправление ошибок}

Дано две строки $A$ и $B$. Надо сказать, сколько есть строк $C$, что из неё можно получить и $A$ и $B$ удалением одного символа.

\section{Игра}

Один ходит влево-вверх, другой ещё и по диагонали. Надо определить, кто первый дойдёт до начала координат.

\section{Куча куч}

Надо определить для каждого $k$ в скольких вершинах нарушится свойство кучи.

\section{Фотография}

Каждый либо $a \times b$, либо $b \times a$. Надо определить $\sum{w} \max{h}$

\section{•}

\end{document}
