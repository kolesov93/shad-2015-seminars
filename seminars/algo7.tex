\documentclass[addpoints]{exam}
\usepackage[utf8]{inputenc}
\usepackage[russian]{babel}
\usepackage[OT1]{fontenc}
\usepackage{amsmath}
\usepackage{amsfonts}
\usepackage{amssymb}
\usepackage{graphicx}
\usepackage{hyperref}
\title{Семинар 7}
\author{Минский ШАД. Весна}

\DeclareMathOperator{\ord}{ord}
\DeclareMathOperator{\suf}{suf}
\DeclareMathOperator{\uniq}{uniq}
\DeclareMathOperator{\cost}{cost}
\DeclareMathOperator{\suc}{suc}


\begin{document}

\maketitle

\section{Двудольность}

Дан двудольный $(n,m)$-граф. Необходимо покрасить его вершины в белый и чёрный цвет, чтоб никакие две смежные вершины не были одного цвета. Стоимость раскраски~--- $C_w \times C_b$, где $C_w$~--- количество белых вершин, а $C_b$~--- количество чёрных.

\begin{itemize}
\item Предложите алгоритм нахождения максимальной по стоимости раскраски за $\mathcal{O}(n + m)$
\item Предложите алгоритм нахождения минимальной по стоимости раскраски за $\mathcal{O}(n^2)$
\item Предложите алгоритм нахождения минимальной по стоимости раскраски за $\mathcal{O}(n \sqrt{n} + m)$

\end{itemize}

\section{Динамическая связность оффлайн}

Дан изначально пустой граф на $n$ вершинах. Также оффлайн даны $q$ запросов одного из трёх типов:

\begin{enumerate}
\item Добавить в граф ребро $(a,b)$
\item Удалить в графе уже добавленное ребро $(a,b)$
\item Проверить, есть ли в графе путь между вершинами $a$ и $b$
\end{enumerate}

Необходимо ответить на все запросы за время $\mathcal{O}(m \sqrt{m} + n)$

\section{Трон}

Дан набор из $n$ точек. Каждая точка задана четырьмя числами $x_i, y_i, dx_i, dy_i$. Точка расположена в координатах $(x_i,y_i)$. Мы можем задать точке вектор движения $(dx_i, dy_i)$ либо $(-dx_i, -dy_i)$. После задания векторов точки начинают движения, причём каждая точка за секунду проходит ровно одну условную единицу расстояния направлено своему вектору движения. Действо останавливается, когда впервые пересекутся две траектории движения.

\begin{itemize}
\item Предложить алгоритм для нахождения самого короткого действа за $\mathcal{O}(n^2)$
\item Предложить алгоритм для нахождения самого длинного действа за $\mathcal{O}(n^2 \log{n})$
\end{itemize}

\section{Ой ли ррр}

Дан алфавит $\Sigma$, причём $|\Sigma| = n$. Составить самую короткую такую строку, что в качестве её подстроки встречаются все $n^2$ упорядоченных пар $(c_1, c_2)$, $c_1, c_2 \in \Sigma$. Например для алфавита $\Sigma = \{a,b,c\}$ такая строка <<aabbccacba>>.

\textbf{Бонус}: найти минимальную лексикографически такую строку

\end{document}
