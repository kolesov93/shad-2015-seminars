\documentclass[addpoints]{exam}
\usepackage[utf8]{inputenc}
\usepackage[russian]{babel}
\usepackage[OT1]{fontenc}
\usepackage{amsmath}
\usepackage{amsfonts}
\usepackage{amssymb}
\usepackage{graphicx}
\usepackage{hyperref}
\title{Семинар 11}
\author{Минский ШАД. Весна}


\begin{document}

\maketitle

\section{Не <<опять>>, а <<снова>>}

Дан массив из $3n + 2$ целых чисел. Уникальных чисел в этом массиве ровно $n+1$, причём $n$ среди них встречаются ровно по $3$ раза, а одно число~--- ровно два раза. Вам нужно детерминировано найти это число за $\mathcal{O}(n)$.

\section{Манька дома~--- Ваньки нет}

У вас есть реализованная структура Queue, которая выполняет операции <<popfront>> и <<pushback>> за $\mathcal{O}(1)$. Вам необходимо, используя один или несколько очередей, реализовать стуктуру Stack, которая позволяет делать операции <<pushback>>, <<popback>>. Постарайтесь оптимизировать одну из операций, если не получается обе сразу. 

\section{Пул}

Дан прямоугольник с целыми измерениями $w \times h$. Он расположен левым нижним углом в точке $(0,0)$, а правым верхним~--- в точке $(w,h)$. В точках $(x_1,y_1)$ и $(x_2,y_2)$ строго внутри прямоугольника расположены два шара (их можно считать точками). В момент времени $0$ они начинают своё движение по вектору $(1,1)$. Такой вектор они проходят за единицу времени. При столкновении со стенкой прямоугольника они меняют свой вектор движения по правилу <<угол падения равен углу отражения>>. Найдите минимальный момент времени, когда два шара встретятся или скажите, что такого момента не существует. Алгоритм должен иметь сложность $\overline{o}(n^{\varepsilon})$ для любого $\varepsilon$.

\section{Без названия}

Дан массив $a$ из $n$ целых неотрицательных чисел. Необходимо найти подотрезок с максимальной $\bigoplus$-суммой. Сложность алгоритм должна составлять $\mathcal{O}(n \log{\max\limits_i{a_i}})$

\section{Чётное деление}

Дан $(n,m)$-граф. Необходимо покрасить все его вершины в два цвета таким образом, чтоб каждая белая вершина была связана с чётным числом белых вершин, а каждая чёрная~--- с чёрным числом чёрных.

\section{Слабое звено}

Дано дерево на $n$ вершинах. После препроцесса необходимо как можно быстрей отвечать на запросы о самом тяжёлом ребре дерева на пути от $A$ до $B$.

\end{document}
