\documentclass[10pt,a4paper]{article}
\usepackage[utf8]{inputenc}
\usepackage[russian]{babel}
\usepackage[OT1]{fontenc}
\usepackage{amsmath}
\usepackage{amsfonts}
\usepackage{amssymb}
\usepackage{graphicx}
\author{Колесов Алексей}
\begin{document}

Вводим пространство $A = (0,1)^{|E|}$

Вводим подпространство $B$, такое что $\forall x \in V$ $\sum\limits_{e \in G(x)} x_e = 0 (\mod 2)$.

Почему это подпространство? Потому что однородное уравнения (значит сумма, кратные тоже лежат внутри, ноль есть).

Чему равна размерность? Пока непонятно. Ответим позже.

Это подпростанство Эйлеровых подмножеств рёбер (так как степень каждой вершины чётная).

Предположим, что граф связный. Научимся быстро находить какой-нибудь элемент множества. Найдём какое-нибудь остовное дерево.

Бывают рёбра дерева, а бывает не рёбра дерева. Можно задать на рёбрах не дерева любое значение, а на остальных определится легко. Свободных переменных $|E| - |V| + 1$. 1 - число компонент связсности.

Возьмём лист. Приходит ровно одно рёбро дерева. Однозначно выражается через остальные.

Любое уравнение получается как сумма остальных.
Выделим любое уравнение. Сложим остальные. Можно видеть, что если ребро не инцидентно вершине, то переменная про него войдёт два раза и сократится. Останутся только наши. Поэтому если выполнили всё, то на последнем ребре ничего не испортится.

Если $e$ - мост, то $P[x_e=1] = 0$. Если $e$ - не мост, то вероятность $0.5$.

Чётность рёбер по любому разрезу равна нулю. Либо просуммируем все уравнения, либо каждый цикл чётное число раз проходит.

Если $e_1,e_2$~--- 2-cut, то $P[x_1 = x_2] = 1$, иначе в половине случаев.

Давайте поисследуем другую конструкцию. Пусть $z$~--- фиксированный вектор из $A$. С какой вероятностью $x \cdot z = 0$. Ответ либо 1, либо $1/2$. Понятно, что если из ортогонального дополнения, то 1. Докажем, что иначе одна вторая.

Пусть $z$ не из ортогонального дополнения. Тогда $\exists x_0$, $(x_0, z) \neq 0$, т.е равен 1.
Окей, давайте разобьём всё пространство на пары $(x, x + x_0)$. Заметим, что это действительно разбиение на пары (если прибавить ко второй компонененте $x_0$, то получим опять $x$. Более того, ровно один из них при умножении на $z$ даст ноль, другой единицу.

Вообще говоря, зачем нам скалярные произведения. Ну нам интересно, что если мы домножим на вектор $(0,0,1,0,0,0,1)$, то мы как раз узнаем, правда ли, что две компоненты одинаковые. Давайте будем исследовать вектора, где есть лишь две единицы.

Оказывается, что такой вектор лежит в ортогональном дополнении, если и только если соответствующие рёбра образуют 2-разрез.

Хотя просто покажем структуру ортогонального подпросранства пространства Cyc. Оказывается, это пространство разрезов. Т.е. если мы поделим вершины на два множества, и возьмём рёбра между разрезами, то это оно.

Давайте проверим, что это пространство. Нужен ноль. Ну, понятно, что если одна из частей - $V$, то ноль есть. Надо проверить, что сумма двух разрезов - разрез. Один $(X,V-X)$, другой $(Y,V-Y) $

Отлично, это пространство. Давайте проверим, что это ещё и то, что нам нужно. То, что оно ортогонально - ну ясно, ведь по разрезу сумма равна нулю. Давайте проверим, что это все векторы.

Посмотрим на размерность пространства циклов. Она $E - V + 1$. А размерность разрезов? Ну $V-1$. Сумма должна быть $E$, хотя это не прямая сумма.

Давайте докажем, что у Cut $V-1$. Давайте просто зафигачим $V-1$ линейно независимых чуваков. Построим остовное дерево. Порежем по каждому ребру. Теперь в таком разрезе есть единичка в этой компоненте, а в других нет.

\end{document}