\documentclass[addpoints]{exam}
\usepackage[utf8]{inputenc}
\usepackage[russian]{babel}
\usepackage[OT1]{fontenc}
\usepackage{amsmath}
\usepackage{amsfonts}
\usepackage{amssymb}
\usepackage{graphicx}
\usepackage{hyperref}
\title{Семинар 8}
\author{Минский ШАД. Весна}

\DeclareMathOperator{\ord}{ord}
\DeclareMathOperator{\suf}{suf}
\DeclareMathOperator{\uniq}{uniq}
\DeclareMathOperator{\cost}{cost}
\DeclareMathOperator{\suc}{suc}


\begin{document}

\maketitle

\section{Одномерный художник}

Дан массив из $n$ чисел. Затем идут запросы: поставить во все элементы от $l$ до $r$ число 1 или ноль. Надо после каждого вопроса отвечать, сколько всего существует максимальных по включению подотрезков массива из единиц.

\section{Квантовый имитатор}

Дан массив $1..n$ (только число $n$). Затем идут $Q$ запросов~--- взять подмассив от $l$ до $r$ и развернуть его. Нужно за $\mathcal{O}(Q \sqrt{Q} + n)$ выписать массив после всех запросов.

\section{Топсорт}

Найти количество графов на $n$ вершинах, для которых существует ровно одна топологическая сортировка. Для которых существует ровно 2.

\section{Круговая порука}

Разбить циклический массив на минимальное число частей таким образом, чтоб сумма в каждой части была меньше $a$.

\section{Чёрный ящик}

Загадана перестановка из $n$ (нечётного) числа первых натуральных чисел. Можно $n$ раз спросить xor двух любых. Надо восстановить перестановку.

\section{RMQ + binsearch}

Дан массив. Можем за одну операцию прибавить к подотрезку равных чисел единицу. Надо за минимальное количество действий сравнять элементы.

\end{document}
