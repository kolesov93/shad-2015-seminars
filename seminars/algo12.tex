\documentclass[addpoints]{exam}
\usepackage[utf8]{inputenc}
\usepackage[russian]{babel}
\usepackage[OT1]{fontenc}
\usepackage{amsmath}
\usepackage{amsfonts}
\usepackage{amssymb}
\usepackage{graphicx}
\usepackage{hyperref}
\title{Семинар 12}
\author{Минский ШАД. Весна}

\DeclareMathOperator{\rnd}{random}

\begin{document}

\maketitle

\section{Ровный}

Вам предоставлена функция $\rnd$, которая возвращает вещественное число, равномерно распределённое на $[0,1)$.

На вход алгоритму подаётся $n$ неотрицательных чисел $p_i$, $\sum\limits_{i=1}^n p_i = 1$. Вам разрешается произвести некоторый препроцесс, а после этого нужно отвечать на вопрос <<вернуть случайное число в диапазоне от 1 до $n$>>. Число $i$ должно возвращаться вашей функцией с вероятностью $p_i$.

\begin{enumerate}
\item Препроцесс за $\mathcal{O}(n)$, ответ за $\mathcal{O}(\log{n})$
\item Пусть все вероятности имеют вид $\frac{a_i}{kn}$, где $k$~--- некоторая константа (во всех остальных пунктах это не так). Препроцесс за $\mathcal{O}(n)$, ответ за $\mathcal{O}(1)$
\item Препроцесс за $\mathcal{O}(n \log{n})$, ответ за $\mathcal{O}(1)$ в ожидаемом среднем
\item Препроцесс за $\mathcal{O}(n \log{n})$, ответ за $\mathcal{O}(1)$
\end{enumerate}

\section{Солнечный круг}

Дан неотрицательно взвешенный $(n,m)$-граф, причём известно, что в каждой его компоненте рёбер максимум на единицу больше, чем вершин. Необходимо после препроцесса за $\mathcal{O}(\log{n})$ отвечать длину кратчайшего пути между любыми двумя вершинами.

\section{Флатландия}

Дано $n$ точек на плоскости своими координатами и константа $c$. Между двумя точками есть отрезок, если и только если для них выполняется $|x_1 - x_2| + |y_1 - y_2| \leqslant c$. Необходимо найти количество компонент связности в данном графе.

\section{Цветная задача}

Дано $n$ натуральных чисел, максимум из которых имеет порядок $\mathcal{O}(n)$. Тройка из этих чисел является хорошей, если все три числа в тройке попарно взаимно просты, либо попарно не взаимно просты. Ваша задача посчитать количество хороших троек за $\mathcal{O}(n \log{n})$.

\section{Хорошо}

Есть $n$ пастбищ. За $w_i$ тугриков на $i$-м пастбище можно выкопать колодец. За $P_{i,j}$ тугриков можно между $i$-м и $j$-м пастбищем прорыть канал. Ваша задача за минимальное количество тугриков сделать так, чтоб на каждом пастбище была вода (либо из колодцев, либо существовал путь по каналом до колодца). Сложность должна составлять $\mathcal{O}(n^2)$.

\end{document}
